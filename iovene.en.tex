\documentclass[10pt]{tccv}
\usepackage[english]{babel}

\begin{document}
	\part{Salvatore Iovene}
		\personal
			[www.iovene.com]
			{Riimukallio 5 A 4\newline 02760 Espoo FINLAND}
			{+358 504804026}
			{salvatore@iovene.com}


		\section{In short}
			I am a software engineer with a BSc in Computer Science. I have
			been working since 2004, and programming since the age of nine,
			when I was gifted with my first computer in 1990.\\
			
			I am a very flexible developer, with experience spanning multiple
			languages and technologies. I am very apt (and most often eager) to
			learn new things and embark in new exciting projects.\\
			
			I consider myself a "maker", and someone hopelessly destined to
			constantly fidget with ideas and projects.


		\section{Technologies I like}
			\begin{factlist}
				\item{Languages}
					{Python, JavaScript, C/C++, Bash,
					HTML/CSS}

				\item{Frameworks}
					{Django, Angular, jQuery, GTK, Qt}

				\item{Misc}
					{Linux, ViM, git }
			\end{factlist}

			\subsection{Please note}
				\emph{There is a lot I have omitted from this section, for
				brevity. It is not a list of technologies I know or have worked
				with, but of things I like and would like to continuing working
				with in the future.}


		\section{Work experience}
				\textbf{May 2011 -- Present // Intel, Espoo (Finland)}
				As as software engineer at the Intel Open Source Technology
				Center, I have been involved with a variety of projects
				spanning multiple technologies. Having been hired as part of a
				move to acquire talent leaving Nokia after the Microsoft deal
				in early 2011, before things settled I have been contributing
				some patches to the
				\href{https://syncevolution.org/}{SyncEvolution} project.\\

				Shortly thereafter, our organization's goals where shifted
				towards web technologies. Some of the projects I have worked on
				include a JS performance analysis framework, a UI toolkit,
				contributions to the \href{http://tizen.org}{Tizen} port of
				\href{http://cordova.apache.org/}{Cordova platform}, a
				contribution to Chromium, and the implementation of the W3C NFC
				API.\\
				
				At this job, I have deepened and perfected my knowledge of the
				HTML5 space and ecosystem. I dabbled with JavaScript,
				CoffeeScript, CSS, SCSS, Less, and lots of tools around them,
				such as unit testing frameworks and tools (Karma, Jasmine,
				PhantomJS), MVC libraries (Angular, Ember, KnockOut, Backbone),
				packaging and deployment solutions (grunt, gulp, npm, bower).

			\subsection{}
				\textbf{Nov 2010 -- Present // AstroBin, home based}
				I have been running a moderately active pet project since late
				2010, a website called \href{http://astrobin.com}{AstroBin}.
				It is an image hosting service and community specifically
				targeted to astrophotographers. I maintain my own virtual
				server and host images on Amazon S3. The website is written in
				Python and Django. It handles over 1,500,000 page views and
				serves 300 terabytes monthly. It features advanced search
				powered by Apache Solr, asynchronous tasks powered by Celery,
				an easy to deploy system powered by Vagrant and virtualenv, and
				is served by nginx and gunicorn.\\

				I have worked on it many nights and weekends, and it has turned
				out to be a great success in the astrophotography community,
				earning me world phame in this niche, and more than one
				invitation to speak at astrophotography conferences.\\

				Thanks to this project I have learned how to manage a
				moderately high load server, how to build a community, how to
				design a visually appealing and usable website, and, among many
				other things, how to deal with the pressure of having an issue
				suddenly make a service unavailable to thousands of people.

			\subsection{}
				\textbf{Jan 2010 -- May 2011 // Nokia, Helsinki (Finland)} At
				Nokia, I have been involved with the design and development of
				key components of the Meego platform, working towards the
				release of the N9 phone. I was part of the real-time
				communications team, focused on the phone and messaging UIs,
				and the middleware components below.\\

				At this job I have mainly been involved with C++ and the Qt/QML
				framework.

			\subsection{}
				\textbf{Mar 2008 -- Dec 2010 // Digia, Helsinki (Finland)} At
				Digia I had a consultancy job for Nokia. I worked on the design
				and development of middleware and UI components for the Maemo
				platform, the Operating System of the N900 phone/Internet
				tablet. Among other things, I designed and developed the event
				logger that tracked calls and messages.\\

				At this job I have acquired a deeper knowledge of C, the
				GObject library and the GTK. I have worked with autotools and
				Debian packages.

			\subsection{}
				\textbf{Sep 2006 -- Mar 2008 // Movial, Helsinki (Finland)} At
				Movial I worked on a C/C++ implementation of the SIP protocol,
				and exposing an API to a video messaging application.


		\section{Education}
			\begin{yearlist}
				\item[University of Salerno, Italy]{2005}
					{BSc in Computer Science, 103/110}
					{Major in Network programming}

				\item[Liceo Scientifico di Scafati, Italy]{1999}
					{Science gymnasium, 97/00}
					{}
			\end{yearlist}


		\section{Public projects}
			\begin{itemize}
				\item
					\href{http://astrobin.com}{AstroBin}, a popular image
					hosting website for astrophotography.

				\item
					\href{https://github.com/siovene/perfectjs}{perfectjs}, a
					comparative performance analysis library for JavaScript.

				\item
					\href{https://review.tizen.org/git/?p=profile/ivi/cowhide.git;a=summary}{Cowhide},
					a UI toolkit built on top of Twitter Bootstrap.

				\item
					Several contributions to the
					\href{http://tizen.org}{Tizen} port of
					\href{http://cordova.apache.org/}{Cordova platform}.

				\item
					A \href{https://codereview.chromium.org/170293004}{modest
					contribution} to the Linux port of the Chromium browser.

				\item
					\href{https://github.com/siovene/crosswalk-nfc-android}{crosswalk-nfc-android},
					an implementation of the W3C NFC specification as a
					\href{http://crosswalk.org}{Crosswalk} extension on
					Android.
				
				\item
					\href{https://github.com/siovene/lannisport}{Lannisport}, a
					theme for the Pelican site generator. 

				\item
					\href{http://maemo.org/api_refs/5.0/5.0-final/eventlogger/index.html}{rtcom-eventlogger},
					a general framework for storing and accessing a persistent
					log.

				\item
					\href{http://www.iovene.com/posts/2010/05/tango-smilies-for-conversations-on-the-nokia-n900/}{conversations-tango-smilies},
					a package that adds support for extra smiley icons on the
					N900 phone.
			\end{itemize}


		\section{Communication skills}
			\begin{factlist}
				\item{Italian}{Native speaker}
				\item{English}{All around fluency}
				\item{Finnish}{Basic skills}
				\item{Spanish}{Survival skills}
			\end{factlist}


		\section{Extracurricula}
			I tend to get involved in multiple projects (software or not) after
			work, and whether this is a point in my favor or not depends on
			your judgement. Here's some of the things I do or have done:

			\begin{itemize}
				\item I am an avid road cyclist: in 2014 I rode eleven thousand
					kilometers.
				\item I am an astrophotographer: with a friend, I own a small
					remote controlled astronomical observatory on a mountain in
					southern Spain, where the weather is close to ideal for
					this sort of things. You can see my work at
					\url{http://astrobin.com/users/siovene/}.
				\item I have authored \href{http://www.geekherocomic.com}{Geek
					Hero Comic}, a humorous webcomic series aboud a software
					developer.
				\item I am a husband and a father.
			\end{itemize}


		\section{About this document}
			This curricum vitae was edited with \LaTeX. You can find the source
			code on GitHub:\\*
			\url{https://github.com/siovene/cv}\\

			You can download an up-to-date PDF at the following URL:\\*
			\url{http://www.iovene.com/pages/cv/}\\
\end{document}
